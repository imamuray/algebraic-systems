\documentclass[uplatex,dvipdfmx,9pt]{beamer}
\usepackage{bxdpx-beamer}  % dvipdfmx 対応
\usepackage{pxjahyper}     % しおりの和文対応
\usepackage[T1]{fontenc}    % フォント: T1
\usepackage{amsmath}       % 数式用
\usepackage[all]{xy}       % 図式
\usepackage{mathtools}
\usepackage{newtxtext, newtxmath}

\renewcommand{\kanjifamilydefault}{\gtdefault} % 和文をゴシックフに
\renewcommand{\familydefault}{\sfdefault}
\usefonttheme[onlymath]{serif}

\usetheme{Antibes}
\usecolortheme{dolphin}
\setbeamertemplate{footline}[frame number] % ページ数表示 
\setbeamertemplate{navigation symbols}  % ナビゲーションシンボル消去

% 数学記号
\newcommand{\defarrow}{\overset{\mathrm{def}}{\Leftrightarrow}}
\newcommand{\nsubgroup}{\vartriangleleft} % 正規部分群
\newcommand{\notnsubgroup}{\ntriangleleft} % 正規部分群の否定
\newcommand{\inverse}[1]{#1^{-1}}
\newcommand{\st}{\text{ s.t. }}
\newcommand{\lt}{\ensuremath <}
\newcommand{\gt}{\ensuremath >}
\newcommand{\generator}[1]{\langle#1\rangle}
\newcommand{\End}{\operatorname{End}}
\newcommand{\Aut}{\operatorname{Aut}}
\newcommand{\Ker}{\operatorname{Ker}}

\newcommand{\sscount}{\textsection \thesubsection}

% 教科書の番号付け
\newcounter{textThmCount}
\newcounter{textLemCount}
\renewcommand{\thetextLemCount}{\Alph{textLemCount}}
\newcounter{textExmCount}

\setbeamertemplate{theorems}[numbered]  % 定理に番号をつける
\theoremstyle{definition} % 斜体にならないようにする
\newtheorem{defn}{Def.}[subsection] % 定義
% \newtheorem{cor}{Cor.}[subsection] % 系
\newtheorem{thm}{Thm.}[subsection] % 定理
\newtheorem{thmText}[textThmCount]{Thm.}
\newtheorem{corText}{Cor.}[textThmCount] % 系
\newtheorem{lemText}[textLemCount]{Lem.} % 補題
\theoremstyle{example}
\newtheorem{exm}{Exm.}[subsection]
\newtheorem{exmText}[textExmCount]{Exm.}

% その他
% Rem. : 注意

% セクション開始毎に目次追加
\AtBeginSubsection[]{
    \begin{frame}
        \tableofcontents[currentsubsection]
    \end{frame}
}

%------------------------

\title{代数系入門}
\subtitle{第3章 環と多項式}
\author{今村勇輝}

\begin{document}
  \begin{frame}[plain]
    \titlepage
  \end{frame}

  % beamer では chapter が使えないので section から始める
  \section{第3章 環と多項式}

    \subsection{\sscount 環とその例}
    \setcounter{textExmCount}{0}

      \begin{frame}

        \begin{defn}
          $R$ : 集合, $R \ne \emptyset$, \\
          $R \times R \to R, (a, b) \mapsto a + b, (a, b) \mapsto ab$
          \begin{enumerate}
            \item $R$ : 加法について可換群
            \item $\forall a, b, c \in R \Rightarrow (ab)c = a(bc)$
            \item $\forall a, b, c \in R \Rightarrow a(b + c) = ab + ac, (b + c)a = ba + ca$
            \item $\exists e \in R \st \forall a \in R, ea = ae = a$
          \end{enumerate}
          $\defarrow$ $R$ : \alert{環}(ring)
        \end{defn}

        \begin{defn}
          $R$: 環 \\
          \begin{itemize}
            \item $\exists! e_+ \in R \st \forall a \in R, e_+ + a = a \defarrow 0 \coloneqq e_+$ : $R$の\alert{零元}
            \item $\forall a \in R, \exists! a' \in R \st a + a' = 0 \defarrow -a \coloneqq a'$
          \end{itemize}
        \end{defn}

        \begin{defn}
          $R$ : 環, $\forall a, b \in R \Rightarrow ab = ba$ $\defarrow$ $R$ : \alert{可換環}
        \end{defn}

      \end{frame}

      \begin{frame}
        
        \begin{thm}
          $R$ : 環, $\exists! e \in R \st \forall a \in R, ea = ae = e$ $\defarrow$ $1 \coloneqq e$ : $R$の\alert{単位元}
        \end{thm}

        \begin{exmText}
          $\mathbb{Z}$ : 可換環 : \alert{有理整数環}
        \end{exmText}

        \begin{exmText}
          $\mathbb{Q}, \mathbb{R}, \mathbb{C}$ : 可換環
        \end{exmText}

        \begin{exmText}
          $[0, 1] \subset \mathbb{R}, R = \{f \mid f\colon [0, 1] \to [0, 1]\},$ \\
          $f, g \in R, \forall t \in [0, 1], (f + g)(t) = f(t) + g(t), (fg)(t) = f(t)g(t)$ $\Rightarrow$ $R$ : 可換環
        \end{exmText}

      \end{frame}

      \begin{frame}

        \begin{exmText}
          $\forall R : \text{環}, \forall S : \text{集合}, S \ne \emptyset, M(S, R) = \{f \mid f\colon S \to R\},$ \\
          $f, g \in M(S, R), \forall x \in S, (f + g)(x) = f(x) + g(x), (fg)(x) = f(x)g(x) \Rightarrow M(S, R) : \text{環}$
        \end{exmText}

        \begin{defn}
          \begin{itemize}
            \item $0 \in M(S, R)$ : $S$から$R$の\alert{零写像}
            \item $-f \in R, \forall x \in S, (-f)(x) = -f(x)$
          \end{itemize}
        \end{defn}
        \underline{Rem.} $\forall x \in S, 0(x) = 0_R$

        \begin{defn}
          $\forall A$ : 加法群, $f\colon A \to A$ : hom. : \alert{自己準同型写像, 自己準同型}(endomorphism) \\
          $\End(A) \coloneqq \{f \mid f\colon A \to A : \text{hom.}\}$
        \end{defn}

        \begin{exmText}
          $\forall A$ : 加法群, \\
          $f, g \in \End(A), \forall x \in A, (f + g)(x) = f(x) + g(x), (fg)(x) = f(g(x)) \Rightarrow \End(A) : \text{環}$
        \end{exmText}
        \underline{Rem.} $\End(A)$ : \alert{自己準同型環}
        
      \end{frame}

      \begin{frame}

        \begin{thm}
          $R$ : 環
          \begin{enumerate}
            \item $0 \in R, \forall a \in R \Rightarrow a0 = 0a = 0$ 
            \item $\forall a, b \in R \Rightarrow a(-b) = (-a)b = -ab$
            \item $\forall a, b \in R \Rightarrow (-a)(-b) = ab$
            \item $\forall a, b, c \in R, b-c \coloneqq b+(-c) \Rightarrow a(b-c) = ab - ac, (b-c)a = ba - ca$
            \item $a_1, \cdots, a_m, b_1, \cdots, b_n \in R \Rightarrow (a_1 + \cdots + a_m)(b_1 + \cdots + b_m) = \displaystyle\sum_{i=1}^m\sum_{j=1}^n a_ib_j$
          \end{enumerate}
        \end{thm}
          
      \end{frame}

    \subsection{\sscount 整域, 体}
    \setcounter{textExmCount}{0}

    \begin{frame}

      \begin{thm}
        $R$ : 環, $0, 1 \in R$ \\
        $1 = 0 \Rightarrow R = \{0\} \; (\because \forall a \in R, a = 1a = 0a = 0)$
      \end{thm}

      \begin{defn}
        $R$ : 環, $0, 1 \in R, 1 = 0$ $\defarrow$ $R$ : \alert{零環}
      \end{defn}
      \underline{Rem.} 今後, $R$は零環ではないとする.

      \begin{defn}
        $\exists a, b \in R \st a \ne 0, b \ne 0, ab = 0 \defarrow a, b : \text{$R$の\alert{零因子}}$ ($a$ : 左零因子, $b$ : 右零因子)
      \end{defn}

      \begin{defn}
        $\forall a, b \in R, a \ne 0, b \ne 0 \Rightarrow ab \ne 0, R : \text{可換}$ $\defarrow$ $R$ : \alert{整域}
      \end{defn}

    \end{frame}

    \begin{frame}

      \begin{exmText}
        $\mathbb{Z}$ : 整域
      \end{exmText}

      \begin{exmText}
        \textsection 1 Exm. 3 は整域ではない
      \end{exmText}

      \begin{defn}
        $a \in R, \exists b \in R \st ba = ab = 1$ $\defarrow$ $a$ : $R$の\alert{可逆元}または\alert{単元}, $\inverse{a} \coloneqq b$ : $a$の\alert{逆元}
      \end{defn}

      \begin{thm}
        \begin{itemize}
          \item $a \in R : \text{単元} \Rightarrow a \ne 0$
          \item $a \in R : \text{単元} \Rightarrow \exists! \inverse{a} \in R \st \inverse{a}a = a\inverse{a} = 1$
        \end{itemize}
      \end{thm}
      
    \end{frame}

    \begin{frame}

      \begin{lemText}
        $R: \text{環}, G = \{a \in R \mid a: \text{$R$の単元}\} \Rightarrow G: \text{乗法に関して群}$
      \end{lemText}

      \begin{exmText}
        $A$ : 加法群, $A \ne \{0\}$
        \begin{itemize}
          \item $f \in \End(A), f: \text{単元} \Rightarrow f: \text{iso.}$
          \item $G = \{f \in \End(A) \mid f: \text{単元}\} \Rightarrow G = \Aut(A)$
        \end{itemize}
      \end{exmText}

      \begin{defn}
        $R$ : 環
        \begin{itemize}
          \item $\forall a \in R, a \ne 0 \Rightarrow a: \text{単元} \defarrow R: \text{\alert{斜体}}$
          \item $R: \text{斜体}, \forall a, b \in R, ab = ba \defarrow R: \text{\alert{体}}$
        \end{itemize}
      \end{defn}

    \end{frame}

    \begin{frame}

      \begin{thm}
        $R$: 環
        \begin{itemize}
          \item $R: \text{斜体} \Leftrightarrow G = \{a \in R \mid a \ne 0\} : \text{乗法に関して群}$
          \item $R: \text{体} \Leftrightarrow G = \{a \in R \mid a \ne 0\} : \text{乗法に関して可換群}$
        \end{itemize}
      \end{thm}
 
      \begin{exmText}
        \begin{itemize}
          \item $\mathbb{Z} : \text{環} \nRightarrow \mathbb{Z} : \text{体}$
          \item $\mathbb{Q, R, C} : \text{環} \Rightarrow \mathbb{Q, R, C} : \text{体}$
        \end{itemize}
      \end{exmText}
      \underline{Rem.} $\mathbb{Q}$ : \alert{有理数体}, $\mathbb{R}$ : \alert{実数体}, $\mathbb{C}$ : \alert{複素数体}

    \end{frame}

    \begin{frame}

      \begin{thm}
        $\forall R : \text{体} \Rightarrow R : \text{整域}$
      \end{thm}

      \begin{lemText}
        $R : \text{整域}, |R| \lt \infty \Rightarrow R : \text{体}$
      \end{lemText}

      \begin{defn}
        $R$ : 環, $R' \subset R, R' \neq \emptyset$ \\
        $R'$ : $R$で定義されている加法, 乗法に関して環, $1_R \in R'$ $\defarrow$ $R'$ : $R$の\alert{部分環}
      \end{defn}

      \begin{thm}
        $R$ : 環, $R' \subset R$ \\
        $R'$ : $R$の部分環 $\Leftrightarrow$ $1_R \in R', \forall a, b \in R' \Rightarrow -a, a+b, ab \in R'$
      \end{thm}

    \end{frame}

    \begin{frame}

      \begin{defn}
        $R'$ : $R$の部分環 \\
        \begin{itemize}
          \item $R'$ : 斜体 $\defarrow$ $R$の\alert{部分斜体}
          \item $R'$ : 体 $\defarrow$ $R$の\alert{部分体}
        \end{itemize}
      \end{defn}

      \begin{exmText}
        \begin{itemize}
          \item 環$\mathbb{Z}$ : 体$\mathbb{Q}$の部分環
          \item 体$\mathbb{Q}$ : 体$\mathbb{R}$の部分体
        \end{itemize}
      \end{exmText}
 
      \begin{exmText}
        $R = \{f \mid f\colon [0,1] \to [0,1]\}$ (\textsection 1 Exm. 3 の環)
        \begin{itemize}
          \item $R' = \{f \mid f\colon [0,1] \to [0,1] : \text{連続関数}\}$ $\Rightarrow$ $R'$ : $R$の部分環
          \item $R'' = \{f \mid f\colon [0,1] \to [0,1] : \text{微分可能関数}\}$ $\Rightarrow$ $R''$ : $R'$の部分環
        \end{itemize}
      \end{exmText}

    \end{frame}

    \begin{frame}

      \begin{defn}
        $R$ : 斜体, $\forall a,b \in R \Rightarrow ab \neq ba$ $\defarrow$ $R$ : \alert{非可換体}
      \end{defn}

      \begin{exmText}
        $\mathbb{C}$ : 複素数の加法群, $A = \mathbb{C} \times \mathbb{C}$
        \begin{enumerate}
          \item $\alpha, \beta \in \mathbb{C}, f_{\alpha, \beta}\colon A \to A, (x,y) \mapsto (\alpha x - \beta y, \bar{\beta} x + \bar{\alpha} y) \Rightarrow f_{\alpha, \beta} \in \End(A)$
          \item $Q = \{f_{\alpha, \beta} \mid \text{上記$f_{\alpha, \beta}$}\}$ $\Rightarrow$ $Q$ : $\End(A)$の部分環
          \item $Q$ : 非可換体
        \end{enumerate}
      \end{exmText}
      \underline{Rem.} $Q$ : $\mathbb{R}$上の\alert{四元数環}

      \begin{thm}
        $R$ : 整域または斜体 $\Rightarrow$ $\exists 0, 1 \in R \st 0 \neq 1$
      \end{thm}
      
    \end{frame}

    \subsection{\sscount イデアルと商環}
    \setcounter{textExmCount}{0}

    \begin{frame}

      \begin{defn}
        $R$ : 環, $J \subset R, J \neq \emptyset$
        \begin{enumerate}
          \item $\forall a, b \in J \Rightarrow a + b \in J$
          \item $\forall r \in R, \forall a \in J \Rightarrow ra \in J$
        \end{enumerate}
        $\defarrow$ $J$ : $R$の\alert{左イデアル}
      \end{defn}

      \begin{thm}
        $R$ : 環, $J \subset R, J \neq \emptyset$ \\
        $J$ : $R$の左イデアル $\Leftrightarrow$ $J \le R$ : 加法部分群, $\forall r \in R, \forall a \in J \Rightarrow ra \in J$
      \end{thm}

      \begin{defn}
        $R$ : 環 \\
        $J \le R$ : 加法部分群, $\forall r \in R, \forall a \in J \Rightarrow ar \in J$ $\defarrow$ $J$ : $R$の\alert{右イデアル}
      \end{defn}

    \end{frame}

    \begin{frame}

      \begin{defn}
        $J$ : $R$の左イデアルかつ右イデアル $\defarrow$ $J$ : $R$の\alert{イデアル}または\alert{両側イデアル}
      \end{defn}
      \underline{Rem.} $R$が可換なら, 左イデアル, 右イデアル, 両側イデアルは一致する
 
      \begin{exmText}
        \hypertarget{exmText3-1}{}
        $R = \{f \mid f\colon [0,1] \to [0,1] : \text{実数値連続関数}\}$ \\
        $c \in [0,1], J_c = \{f \in R \mid f(c) = 0\}$ $\Rightarrow$ $J_c$ : $R$のイデアル
      \end{exmText}

      \begin{exmText}
        $n \in \mathbb{Z}, n \ge 0$ $\Rightarrow$ $n\mathbb{Z}$ : 環$\mathbb{Z}$のイデアル
      \end{exmText}

    \end{frame}

    \begin{frame}

      \begin{exmText}
        $R$ : 環
        \begin{itemize}
          \item $a \in R, J_a = \{xa \mid x \in R\} \Rightarrow J_a : \text{左イデアル}$
          \item $a_1, \cdots, a_n \in R, J = \{x_1a_1 + \cdots + x_na_n \mid x_1, \cdots x_n \in R\} \Rightarrow J : \text{左イデアル}$
        \end{itemize}
      \end{exmText}

      \begin{defn}
        $R$ : 環
        \begin{itemize}
          \item $a \in R, Ra \text{(または$(a)$)} \coloneqq \{xa \mid x \in R\}$ : $a$によって\alert{生成}される\alert{単項左イデアル}
          \item $a_1, \cdots, a_n \in R, (a_1, \cdots, a_n) \coloneqq \{x_1a_1 + \cdots + x_na_n \mid x_1, \cdots, x_n \in R\}$ \\
                : $a_1, \cdots, a_n$によって\alert{生成}される左イデアル
        \end{itemize}
      \end{defn}

      \begin{thm}
        $\forall J : \text{$R$の左イデアル}, a_1, \cdots, a_n \in J \Rightarrow (a_1, \cdots, a_n) \subset J$
      \end{thm}

    \end{frame}

    \begin{frame}

       \begin{thm}
        $R$ : 環
        \begin{itemize}
          \item $1 \in R \Rightarrow (1) = R$
          \item $0 \in R \Rightarrow (0) = \{0\}$
          \item $R, (0)$ : $R$の両側イデアル
        \end{itemize}
      \end{thm}

      \begin{defn}
        $J$ : $R$のイデアル, $J = \{0\}$ $\defarrow$ $0 \coloneqq J$ : \alert{零イデアル}
      \end{defn}

      \begin{thmText}
        $R$ : 環, $R \neq \emptyset$ \\
        $R$ : 斜体 $\Leftrightarrow$ $\forall J$ : $R$の左イデアル $\Rightarrow$ $J = R \; \text{or} \; 0$
      \end{thmText}
      \underline{Rem.} 右イデアルも同様に成り立つが, 両側イデアルの場合 $\Leftarrow$ は必ずしも成り立たない.

    \end{frame}

    \begin{frame}

      \begin{lemText}
        $R$ : 環, $J$ : $R$のイデアル \\
        $\forall a, a', b, b' \in R, a \equiv a' \pmod{J}, b \equiv b' \pmod{J} \Rightarrow ab \equiv a'b' \pmod{J}$
      \end{lemText}

      \begin{thmText}
        $R$ : 環, $J$ : $R$のイデアル, $R/J \ni \bar{a} \coloneqq a + J$ \\
        $\bar{a}, \bar{b} \in R/J, \bar{a} + \bar{b} = \overline{a + b}, \; \bar{a}\bar{b} = \overline{ab}$ $\Rightarrow$ $R/J$ : 環
      \end{thmText}

      \begin{defn}
        $R/J$ : $R$の$J$による\alert{剰余環}または\alert{商環}
      \end{defn}

      \begin{thm}
        \begin{enumerate}
          \item $R/J$ : 零環 $\Leftrightarrow$ $J = R$
          \item $J = (0) \Rightarrow R/J = R$
          \item $R$ : 可換環 $\Rightarrow$ $\forall R/J$ : 可換
        \end{enumerate}
      \end{thm}

    \end{frame}

    \subsection{\sscount \texorpdfstring{ $\mathbb{Z}$}{Z}の商環}
    \setcounter{textExmCount}{0}

    \begin{frame}
      本節では特に有理数環$\mathbb{Z}$について考える.
      
      \begin{thm}
        $n \ge 0, (n) \coloneqq n\mathbb{Z}$ : $\mathbb{Z}$のイデアル ($\because$ \textsection 3 Exm. 2) \\
        $\forall J$ : $\mathbb{Z}$のイデアル $\Leftrightarrow$ $\exists n \in \mathbb{Z} \st n \ge 0, J = (n)$
      \end{thm}

      \begin{defn}
        $n \ge 1, \mathbb{Z}_n \coloneqq \mathbb{Z} / (n)$ : 法$n$に関する$\mathbb{Z}$の商環
      \end{defn}
      \underline{Rem.} $|\mathbb{Z}_n| = n, \mathbb{Z}_1 = \{0\}$

      \begin{defn}
        $n \ge 2, \mathbb{Z}_n \ni \bar{a} \coloneqq a + (n)$
      \end{defn}

    \end{frame}

    \begin{frame}

      \begin{thm}
        $\bar{a} \in \mathbb{Z}_n, \bar{a} \neq \bar{0}, (a, n) = 1 \Rightarrow \forall a' \in a + (n), (a', n) = 1$
      \end{thm}
      \underline{Rem.} $\bar{a}$ : 第1章 \textsection 8 の「法$n$に関する既約剰余類」のこと

      \begin{lemText}
        $n \ge 2, \bar{a} \in \mathbb{Z}_n, \bar{a} \neq \bar{0}$
        \begin{itemize}
          \item $(a, n) = 1 \Rightarrow \bar{a} : \text{unit}$
          \item $(a, n) \neq 1 \Rightarrow \bar{a} : \text{零因子}$
        \end{itemize}
      \end{lemText}

      \begin{thmText}
        $n \ge 2$
        \begin{itemize}
          \item $n = p : \text{素数} \Rightarrow \mathbb{Z}_p : \text{体}$
          \item $n : \text{素数でない} \Rightarrow \exists \bar{a} \in \mathbb{Z}_n \st \bar{a} : \text{零因子}$
        \end{itemize}
      \end{thmText}
      \underline{Rem.} $\mathbb{Z}_2 = \{\bar{0}, \bar{1}\}$ : 体

    \end{frame}

    \begin{frame}
 
      \begin{defn}
        $n \ge 2, G = \{\bar{a} \in \mathbb{Z}_n \mid (a, n) = 1\} \Rightarrow G : \text{群} \; (\because \text{Lem. D})$ \\
        $(\mathbb{Z} / n\mathbb{Z})^\times \coloneqq G$ : 法$n$に関する$\mathbb{Z}$の\alert{既約剰余類群}
      \end{defn}

      \begin{defn}
        $\varphi(n) \coloneqq |\{a \mid 1 \le a \le n, (n, a) = 1\}|$ : \alert{Eulerの関数}
      \end{defn}

      \begin{thm}
        $|(\mathbb{Z} / n\mathbb{Z})^\times| = \varphi(n)$
      \end{thm}

      \begin{thm}[Euler]
        $a, m \in \mathbb{Z}, m \ge 0, (a, m) = 1 \Rightarrow a^{\varphi(m)} \equiv 1 \pmod{m}$
      \end{thm}

    \end{frame}

    \begin{frame}
 
      \begin{thm}
        $p$ : 素数
        \begin{itemize}
          \item $|(\mathbb{Z} / p\mathbb{Z})^\times| = p - 1$
          \item $(\mathbb{Z} / p\mathbb{Z})^\times = \{\bar{a} \in \mathbb{Z}_p \mid \bar{a} \neq \bar{0}\}$
          \item $(\mathbb{Z} / p\mathbb{Z})^\times$ : 巡回群
        \end{itemize}
      \end{thm}
      
    \end{frame}

    \subsection{\sscount 準同型写像}
    \setcounter{textExmCount}{0}

    \begin{frame}

      \begin{defn}
        $R, R'$ : 環, $f\colon R \to R'$ \\
        $f(1_R) = 1_{R'}, \forall x, y \in R, f(x + y) = f(x) + f(y), f(xy) = f(x)f(y)$ $\defarrow$ $f$ : \alert{準同型写像}
      \end{defn}
      \underline{Rem.} 加法群の準同型写像と区別する場合は\alert{環準同型(写像)}とよぶ

      \begin{thm}
        $R, R', R''$ : 環 \\
        $f\colon R \to R' : \text{hom.}, g\colon R' \to R'' : \text{hom.} \Rightarrow g \circ f \colon R \to R'' : \text{hom.}$
      \end{thm}
      \underline{Rem.} \alert{単射準同型}, \alert{全射準同型}, \alert{同型}, \alert{自己同型}などの語の用法は群の場合と同様

      \begin{defn}
        $R, R'$ : 環 \\
        $\exists f\colon R \to R' : \text{iso.}$ $\defarrow$ $R \cong R'$ : $R$と$R'$は\alert{同型}
      \end{defn}
      
    \end{frame}

    \begin{frame}

      \begin{exmText}
        $R = \{f \mid f\colon [0, 1] \to [0, 1] : \text{連続関数}\}$ \\
        $c \in [0, 1], F\colon R \to \mathbb{R}, f \mapsto f(c) \Rightarrow F : \text{hom.}$
      \end{exmText}

      \begin{exmText}
        $f\colon \mathbb{C} \to \mathbb{C}, \alpha \mapsto \bar{\alpha} \Rightarrow f \in \Aut(\mathbb{C})$
      \end{exmText}

      \begin{exmText}
        $R = \{a + b\sqrt{2} \mid a, b \in \mathbb{Z}\}$ : 環 \\
        $f\colon R \to R, a + b\sqrt{2} \mapsto a - b\sqrt{2} \Rightarrow f \in \Aut(R)$
      \end{exmText}
      
    \end{frame}

    \begin{frame}

      \begin{exmText}
        $R$ : 環, $J$ : $R$のイデアル \\
        $\varphi \colon R \to R/J, a \mapsto \bar{a} \Rightarrow \varphi : \text{hom.}$ : \alert{標準的準同型写像}または\alert{自然な準同型写像}
      \end{exmText}

      \begin{thm}
        $R, R'$ : 環 \\
        $f\colon R \to R' : \text{hom.} \Rightarrow f(0_R) = 0_{R'}, f(-x) = -f(x)$
      \end{thm}

      \begin{thm}
        $R, R'$ : 環 \\
        $f\colon R \to R' : \text{hom.} \Rightarrow R' \supset f(R) : \text{環}$
      \end{thm}
      
    \end{frame}

    \begin{frame}
      
      \begin{defn}
        $R, R'$ : 環, $f\colon R \to R'$ : hom. \\
        $\inverse{f}(0) = \{x \in R \mid f(x) = 0\}$ $\defarrow$ $\Ker f \coloneqq \inverse{f}(0)$ : $f$の\alert{核}
      \end{defn}
      \underline{Rem.} $\Ker f$は加法群の準同型としての$f$の核にほかならない

      \begin{thm}
        $R, R'$ : 環 \\
        $f\colon R \to R' : \text{hom.} \Rightarrow \Ker f : \text{$R$の両側イデアル}$
      \end{thm}

      \begin{thm}
        $R, R'$ : 環, $f\colon R \to R' : \text{hom.}$ \\
        $R'= \{0\} \Leftrightarrow \ker f = R$
      \end{thm}

    \end{frame}

    \begin{frame}

      \begin{exmText}
        $R = \{f \mid f\colon [0, 1] \to [0, 1] : \text{連続関数}\}$ \\
        $F\colon f \mapsto f(c)$ : hom. (Exm. 1), $J_c = \{f \in R \mid f(c) = 0\}$ (\hyperlink{exmText3-1}{\textsection 3 Exm. 1}) $\Rightarrow$ $\ker F = J_c$
      \end{exmText}

      \begin{exmText}
        $R$ : 環 \\
        $\forall J$ : $R$のイデアル, $f\colon R \to R/J$ $\Rightarrow$ $\ker f = J$
      \end{exmText}
      
    \end{frame}

    \begin{frame}

      \begin{thm}
        $R$ : 斜体, $R' \neq \emptyset$ : 環, $f\colon R \to R'$ : hom. $\Rightarrow$ $f$ : injective hom.
      \end{thm}

      \begin{thmText}[環の準同型定理]
        $R, R'$ : 環 \\
        $f\colon R \to R' : \text{hom.}, \Ker f = J \Rightarrow g\colon R/J \to f(R) : \text{iso.} \; (R/J \cong f(R))$ 
      \end{thmText}

      \begin{thmText}
         $R, R'$ : 環, $f\colon R \to R'$ : surjective hom., $\Ker f = J$
        \begin{itemize}
          \item $M$ : left ideal of $R$, $M'$ : left ideal of $R'$, $J \subset M$ $\Rightarrow$ $f(M) = M', \inverse{f}(M') = M$
          \item $M$ : right ideal of $R$, $M'$ : right ideal of $R'$, $J \subset M$ $\Rightarrow$ $f(M) = M', \inverse{f}(M') = M$
          \item $M$ : ideal of $R$ $\Leftrightarrow$ $M'$ : ideal of $R$, $R/M \cong R'/M'$
        \end{itemize}       
      \end{thmText}
      
    \end{frame}

    \subsection{\sscount 商の体}
    \setcounter{textExmCount}{0}

    \subsection{\sscount 多項式環}
    \setcounter{textExmCount}{0}

    \subsection{\sscount 体の上の多項式, 単項イデアル整域}
    \setcounter{textExmCount}{0}

    \subsection{\sscount 素元分解とその一意性}
    \setcounter{textExmCount}{0}

    \subsection{\sscount \texorpdfstring{ $\mathbb{Z}[i]$}{Z[i]}の素元}
    \setcounter{textExmCount}{0}

    \subsection{\sscount 多項式の根, 代数的閉体}
    \setcounter{textExmCount}{0}

    \subsection{\sscount \texorpdfstring{ $\mathbb{Z}$または$\mathbb{Q}$}{ZまたはQ}の上の多項式}
    \setcounter{textExmCount}{0}

    \subsection{\sscount 多変数の多項式}
    \setcounter{textExmCount}{0}

\end{document}