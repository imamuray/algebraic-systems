\documentclass[uplatex,dvipdfmx,9pt]{beamer}
\usepackage{bxdpx-beamer}  % dvipdfmx 対応
\usepackage{pxjahyper}     % しおりの和文対応
\usepackage[T1]{fontenc}    % フォント: T1
\usepackage{amsmath}       % 数式用
\usepackage[all]{xy}       % 図式
\usepackage{mathtools}
\usepackage{newtxtext, newtxmath}

\renewcommand{\kanjifamilydefault}{\gtdefault} % 和文をゴシックフに
\renewcommand{\familydefault}{\sfdefault}
\usefonttheme[onlymath]{serif}

\usetheme{Antibes}
\usecolortheme{dolphin}
\setbeamertemplate{footline}[frame number] % ページ数表示 
\setbeamertemplate{navigation symbols}  % ナビゲーションシンボル消去

% 数学記号
\newcommand{\defarrow}{\overset{\mathrm{def}}{\Leftrightarrow}}
\newcommand{\nsubgroup}{\vartriangleleft} % 正規部分群
\newcommand{\notnsubgroup}{\ntriangleleft} % 正規部分群の否定
\newcommand{\inverse}[1]{#1^{-1}}
\newcommand{\st}{\text{ s.t. }}
\newcommand{\lt}{\ensuremath <}
\newcommand{\gt}{\ensuremath >}
\newcommand{\generator}[1]{\langle#1\rangle}

\newcommand{\sscount}{\textsection \thesubsection}

% 教科書の番号付け
\newcounter{textThmCount}
\newcounter{textLemCount}
\renewcommand{\thetextLemCount}{\Alph{textLemCount}}
\newcounter{textExmCount}

\setbeamertemplate{theorems}[numbered]  % 定理に番号をつける
\theoremstyle{definition} % 斜体にならないようにする
\newtheorem{defn}{Def.}[subsection] % 定義
% \newtheorem{cor}{Cor.}[subsection] % 系
\newtheorem{thm}{Thm.}[subsection] % 定理
\newtheorem{thmText}[textThmCount]{Thm.}
\newtheorem{corText}{Cor.}[textThmCount] % 系
\newtheorem{lemText}[textLemCount]{Lem.} % 補題
\theoremstyle{example}
\newtheorem{exm}{Exm.}[subsection]
\newtheorem{exmText}[textExmCount]{Exm.}

% その他
% Rem. : 注意

% セクション開始毎に目次追加
\AtBeginSubsection[]{
    \begin{frame}
        \tableofcontents[currentsubsection]
    \end{frame}
}

%------------------------

\title{代数系入門}
\subtitle{第3章 環と多項式}
\author{今村勇輝}

\begin{document}
  \begin{frame}[plain]
    \titlepage
  \end{frame}

  % beamer では chapter が使えないので section から始める
  \section{第3章 環と多項式}

    \subsection{\sscount 環とその例}
    \setcounter{textExmCount}{0}

      \begin{frame}

        \begin{defn}
          $R$ : 集合, $R \ne \emptyset$, \\
          $R \times R \to R, (a, b) \mapsto a + b, (a, b) \mapsto ab$
          \begin{enumerate}
            \item $R$ : 加法について可換群
            \item $\forall a, b, c \in R \Rightarrow (ab)c = a(bc)$
            \item $\forall a, b, c, d \in R \Rightarrow a(b + c) = ab + ac, (b + c)a = ba + ca$
            \item $\exists e \in R \st \forall a \in R, ea = ae = a$
          \end{enumerate}
          $\defarrow$ $R$ : \alert{環}(ring)
        \end{defn}

        \begin{defn}
          $R$: 環 \\
          \begin{itemize}
            \item $\exists! e_+ \in R \st \forall a \in R, e_+ + a = a \defarrow 0 \coloneqq e_+$ : $R$の\alert{零元}
            \item $\exists! a' \in R \st a \in R, a + a' = 0 \defarrow -a \coloneqq a'$
          \end{itemize}
        \end{defn}

        \begin{defn}
          $R$ : 環, $\forall a, b \in R \Rightarrow ab = ba$ $\defarrow$ $R$ : \alert{可換環}
        \end{defn}

      \end{frame}

      \begin{frame}
        
        \begin{thm}
          $R$ : 環, $\exists! e \in R \st \forall a \in R, ea = ae = e$ $\defarrow$ $1 \coloneqq e$ : $R$の\alert{単位元}
        \end{thm}

        \begin{exmText}
          $\mathbb{Z}$ : 可換環 : \alert{有理整数環}
        \end{exmText}

        \begin{exmText}
          $\mathbb{Q}, \mathbb{R}, \mathbb{C}$ : 可換環
        \end{exmText}

        \begin{exmText}
          $[0, 1] \subset \mathbb{R}, R = \{f \mid f\colon [0, 1] \to [0, 1]\},$ \\
          $f, g \in R, \forall t \in [0, 1], (f + g)(t) = f(t) + g(t), (fg)(t) = f(t)g(t)$ $\Rightarrow$ $R$ : 可換環
        \end{exmText}

      \end{frame}

      \begin{frame}

        \begin{exmText}
          $\forall R : \text{環}, \forall S : \text{集合}, S \ne \emptyset, M(S, R) = \{f \mid f\colon S \to R\},$ \\
          $f, g \in M(S, R), \forall x \in S, (f + g)(x) = f(x) + g(x), (fg)(x) = f(x)g(x) \Rightarrow M(S, R) : \text{環}$
        \end{exmText}

        \begin{defn}
          $0 \in M(S, R)$ : $S$から$R$の\alert{零写像} \\
        \end{defn}
        \underline{Rem.} $\forall x \in S, 0(x) = 0_R$

        \begin{defn}
          $\forall A$ : 加法群, $f\colon A \to A$ : hom. : \alert{自己準同型写像, 自己準同型}(endomorphism) \\
          $End(A) \coloneqq \{f \mid f\colon A \to A : \text{hom.}\}$
        \end{defn}

        \begin{exmText}
          $\forall A$ : 加法群, \\
          $f, g \in End(A), \forall x \in A, (f + g)(x) = f(x) + g(x), (fg)(x) = f(x)g(x) \Rightarrow End(A) : \text{環}$
        \end{exmText}
        \underline{Rem.} $End(A)$ : \alert{自己準同型環}
        
      \end{frame}

      \begin{frame}

        \begin{thm}
          $R$ : 環
          \begin{enumerate}
            \item $0 \in R, \forall a \in R \Rightarrow a0 = 0a = 0$ 
            \item $\forall a, b \in R \Rightarrow a(-b) = (-a)b = -ab$
            \item $\forall a, b \in R \Rightarrow (-a)(-b) = ab$
            \item $\forall a, b, c \in R \Rightarrow a(b-c) = ab - ac, (b-c)a = ba - ca$
            \item $a_1, \cdots, a_m, b_1, \cdots, b_n \in R \Rightarrow (a_1 + \cdots + a_m)(b_1 + \cdots + b_m) = \displaystyle\sum_{i=1}^m\sum_{j=1}^n a_ib_j$
          \end{enumerate}
        \end{thm}
          
      \end{frame}

    \subsection{\sscount 整域, 体}
    \setcounter{textExmCount}{0}

    \subsection{\sscount イデアルと商環}
    \setcounter{textExmCount}{0}

    \subsection{\sscount \texorpdfstring{ $\mathbb{Z}$}{Z}の商環}
    \setcounter{textExmCount}{0}

    \subsection{\sscount 準同型写像}
    \setcounter{textExmCount}{0}

    \subsection{\sscount 商の体}
    \setcounter{textExmCount}{0}

    \subsection{\sscount 多項式環}
    \setcounter{textExmCount}{0}

    \subsection{\sscount 体の上の多項式, 単項イデアル整域}
    \setcounter{textExmCount}{0}

    \subsection{\sscount 素元分解とその一意性}
    \setcounter{textExmCount}{0}

    \subsection{\sscount \texorpdfstring{ $\mathbb{Z}[i]$}{Z[i]}の素元}
    \setcounter{textExmCount}{0}

    \subsection{\sscount 多項式の根, 代数的閉体}
    \setcounter{textExmCount}{0}

    \subsection{\sscount \texorpdfstring{ $\mathbb{Z}$または$\mathbb{Q}$}{ZまたはQ}の上の多項式}
    \setcounter{textExmCount}{0}

    \subsection{\sscount 多変数の多項式}
    \setcounter{textExmCount}{0}

\end{document}